\chapter*{Wstęp}
\label{ch:wstep}
\addcontentsline{toc}{chapter}{Wstęp}
Poniższa praca przedstawia problemy z którymi mierzą się współczesne firmy oraz ich rozwiązania, w celu stworzenia nowych, wydajnych a także intuicyjnych produktów, w tym kontrolerów dla użytkowników świata wirtualnego. Kontrolery w szczególności napotykają wiele problemów ze względu na przeszkody wynikające z natury nawigacji inercyjnej, czyli określenia położenia i orientacji zazwyczaj w środowisku bez punktu odniesienia. Problemy te zostaną opisane w rozdziale~\ref{ch:komponenty}, a także metody jakie zostały zastosowane aby rozwiązać wspomniane problemy przez liderów branży.

Z biegiem czasu dąży się do jak najbardziej realistycznych rozwiązań, tak aby w pełni oddać naturę świata rzeczywistego w środowisku programowalnym. W tym celu powstały pierwsze komercyjne modele rękawic, pozwalających na ruch dłoni w przestrzeni świata wirtualnego, co dało możliwość stworzenia naturalnego interfejsu pomiędzy użytkownikiem a światem wirtualnym w którym się znajduje. Naturalność tego rozwiązania, oraz potencjał jaki w sobie kryje, sprawiło że wiele firm zaczęło tworzyć własne rozwiązania tego produktu, dodając unikalne właściwości takie jak wibracje symulujące dotyk, punktową imitacje nacisku czy też szkielet blokujący palce dłoni przed zgięciem. Rozwój tego rynku oraz chęć inwestycji w tego typu projekty ze strony dużych firm i organizacji pokazuje zapotrzebowanie na tego typu rozwiązanie. Potencjał który się w tym kryje, oraz brak uniwersalnego rozwiązania jest głównym powodem dla powstania tej pracy.

Celem tej pracy jest stworzenie własnego, uproszczonego kontrolera-rękawicy, w celu lepszego poznania znanych problemów dla tego typu urządzeń oraz własnej implementacji ich rozwiązań. Kontroler ten powinien być prosty w wykonaniu, tak aby można było go odtworzyć w domowych warunkach przy jednoczesnym spełnieniu podstawowych wymagań, takich jak określenie orientacji, położenia oraz momentu zgięcia poszczególnych palców. Oprócz tego, aby umożliwić interakcję z kontrolerem zostanie zaprezentowany proces tworzenia aplikacji na system Android, która będzie łączyć się z rękawicą i prezentować przesłane dane. Na podstawie tych informacji zostanie wyświetlony w czasie rzeczywistym model dłoni na ekranie smart-fona, który będzie reagował na zmiany w przesyłanych danych.