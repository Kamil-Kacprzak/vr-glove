\chapter{Podsumowanie}
\label{ch:podsumowanie}
Przedmiotem działań podjętych w celu stworzenia tej pracy było stworzenie podstawowej wersji kontrolera którego można używać w sposób intuicyjny poprzez samo poruszanie dłonią, w szczególności przeznaczonych do środowisk wirtualnej rzeczywistości. Projekt miał za zadanie przygotowanie podłoża dla dalszych prac, usprawnień a także poznania przykładowych rozwiązań znajdujących się na rynku. Podjęto działanie w celu zrozumienia problemów oraz rozwiązań nawigacji inercyjnej przy użyciu IMU, poznając zasady działania sensorów które jednostka ta obsługuje oraz standardu przesyłania tych danych poprzez adapter Bluetooth Low Energy. Został również zaadresowany problem pomiaru wygięcia palców przy niewielkim budżecie. Wszystkie te elementy zostały złożone w całość tworząc w pełni sprawny kontroler, zdolny do realizacji przedstawionych celów. Kontroler ten niestety ma też wiele wad które sprawiają że wymaga on wielu usprawnień w przyszłości. W celu poznania metod obsługi kontrolera została napisana aplikacja na system Android, pozwalająca wykorzystać stworzone urządzenie w praktyce. Wykorzystując SDK \textit{Sceneform} została przedstawiony model dłoni, reagujący na działanie kontrolera. Dedykowana aplikacja prezentuje wizualną wersję problemów nawigacji inercyjnej, pokazując skalę problemów które trzeba rozwiązać aby osiągnąć satysfakcjonujące efekty. Dzięki zebranym danym, ustalono błędy wykonane w projekcie oraz przedstawiono przykładowe sposoby na ich rozwiązanie bądź usprawnienie.  

Projekt ten pozwolił na wyciągnięcie wielu cennych lekcji dzięki zrozumieniu wszystkich jego elementów. Dzięki temu można stwierdzić że jest on niejako prototypem docelowego projektu, któremu należy poświęcić dużo więcej pracy. Mając jednak do dyspozycji odpowiednie narzędzia i budżet można osiągnąć satysfakcjonujące rozwiązanie nawet w domowych warunkach w szczególności jeżeli  poruszanym tematem jest orientacja a także kształt dłoni w czasie rzeczywistym. W celu osiągnięcia dokładnego położenia, najlepszym rozwiązaniem pozostaje jak dotąd użycie zewnętrznego systemu do śledzenia położenia. Jeżeli chcemy skorzystać z tego projektu przy użyciu dostępnych na rynku okularów VR, które posiadają w zestawie system namierzania kontrolerów, należałoby jako dodatkowy element zapewnić synchronizację, tworząc tym samym kompletny produkt w domowych warunkach. Jest to niewątpliwie projekt o dużym potencjale, z wciąż rosnącym rynkiem oraz zapotrzebowaniem i zainteresowaniem na tego typu produkty pokazywanym przez wielkie koncerny samochodowe a nawet agencje kosmiczne. Na zakończenie warto dodać że wciąż nie odkryto pełnych możliwości wirtualnej rzeczywistości, co sprawia że praca nad tym projektem jest tak ekscytująca, a na przykładzie tego projektu pokazano że nie jest to ekskluzywny rynek i każdy może w nim wziąć udział w celu zbudowania lepszej i być może wirtualnej przyszłości.  