\begin{abstract}
Tematem pracy jest ``Rozwój technologii wirtualnej rzeczywistości na przykładzie rękawicy-kontrolera" w której pokazano czym jest świat wirtualny a także w jaki sposób kontroluje się w nim środowisko. Przedstawiono tematykę kontrolerów a także rękawicy-kontrolera, oraz ich podstawowych funkcjonalności. W rozdziale~\ref{ch:komponenty} została pokazane problemy oraz rozwiązania z jakimi się mierzą współczesne rozwiązania. Następnie pokazano własną implementacje rękawicy-kontrolera spełniającej podstawowe funkcje. Rękawica ta przesyła dane bezprzewodowo. Przykładowy odbiornik tych danych, w postaci aplikacji na Androida został pokazany w rozdziale~\ref{ch:aplikacja}. Rozdział ten porusza takie tematy jako implementacja łączności bluetooth, ustalenie orientacji, położenia a także animacja modelu dłoni na podstawie otrzymanych danych z kontrolera. Na podstawie pracy z projektem zostały wyciągnięte wnioski oraz możliwe ulepszenia projektu pokazane w rozdziale~\ref{ch:rozwoj}, po czym zostało sformułowane podsumowanie przeprowadzonego projektu, pokazujące trudności na jakie się napotkano w trakcie pracy. Problemy te należą do dziedziny nawigacji inercyjnej i nie istnieje uniwersalne rozwiązanie które można by zastosować. Celem pracy jest przeprowadzanie próby stworzenia własnego rozwiązania realizującego zagadnienia związane z tematyką kontrolerów - w tym przypadku prezentowanego jako rękawica.
\\
\\Słowa kluczowe: Wirtualna rzeczywistość, Kontroler,Rękawica, Nawigacja inercyjna, Akcelerometr, Żyroskop, Czujnik wygięcia
\newpage
\end{abstract}
\selectlanguage{english} 
\begin{abstract}
The topic of this thesis is the "Development of virtual reality based on the example of a glove controller". This thesis presents (the concept of) virtual reality as well as ways of controlling and interacting within its universe. The topic of controller, especially in the context of glove controllers, was addressed and its functionality and usability cases were shown. Chapter ~\ref{ch:komponenty} highlights the challenges encountered by the controller’s designers. Next, the implementation of the glove controller is presented, with a focus on delivering basic functionality. Data is sent from the glove in a wireless manner. The receiver of this data, an Android application, is described in chapter~\ref{ch:aplikacja}. The mentioned chapter discusses topics of Bluetooth Low energy implementation, the orientation of the device, its displacement and the model animation based on data received from the glove. On the basis of the experiences acquired while working on this project, flaws and possible improvements were outlined in chapter~\ref{ch:rozwoj}, followed by a summary and multiple conclusions formed in relation to the project, detailing the variety of obstacles faced while creating the glove proposal. The issues encountered relate to the immersive navigation realm, where no universal solution could be applied. The goal of this paper is to attempt creating a homemade glove controller, which fulfils all of the basic requirements for a functional controller device.
\\
\\Key words: Virtual reality, Controller, Glove, Immersive navigation, Accelerometer, Gyroscope, Bend sensors 
\end{abstract}
 