\chapter{Dalszy rozwój projektu}
\label{ch:rozwoj}
 W rozdziałach~\ref{ch:rekawica} i~\ref{ch:aplikacja} pokazano budowę kontrolera oraz aplikacji która wykorzystuje zbudowany kontroler w celu obsługi podstawowych funkcji rękawicy-kontrolera. Projekt ten powstał z myślą ograniczonego budżetu, prostoty wykonania oraz możliwości replikacji. Założenia te spowodowały że zdecydowano się na pewne rozwiązania które w końcowej wersji projektu pokazały swoje wady. W tym rozdziale zostanie poruszony temat błędów popełnionych w pierwszej wersji tego projektu oraz przykładowe sposoby na ich rozwiązanie w przyszłości. 
 
 \section{Problemy mikroprocesora}
 \label{sec:iuMikroprocesor}
 Przede wszystkim szukano małego mikrokontrolera tak aby nie był on przeszkodą podczas użytkowania kontrolera. O ile założenie to było dobrym pomysłem, okazało się że umiejscowienie przy brzegu sprawiło że ruch palców, w szczególności kciuka, może zmienić położenie jednostki IMU na rękawicy. Oznacza to że nawet jeśli nasza ręka znajduję się w stałej pozycji, samo poruszanie palcami wprowadza błąd w odczycie. Niestety wybór tego produktu od Arduino również przysporzył wiele kłopotów z racji błędnego rozłączania się adaptera Bluetooth. Z dotychczasowego użytkowania można stwierdzić iż kontroler poprawnie łączy się i rozłącza dwa razy, zaś w większości testowanych przypadków dochodzi do błędu połączenia przy trzeciej próbie. Aby usunąć ten błąd należy odłączyć płytkę od zasilania i podłączyć ponownie, resetując tym samym moduł. Samo oprogramowanie rękawicy skupia się na odczytach z dwóch sensorów. Praktyka pokazała że dane te nie są wystarczająco dokładne, i jeżeli to możliwe powinny być pobierane również dane magnetometru w celu dodatkowego korygowania odczytów z żyroskopu. 
 
 \section{Problemy budowy i czujników wygięcia}
 \label{sec:iuPalce}
 Problem z użytkowaniem rękawicy pojawił się dość szybko od jej zbudowania. Mianowicie wybrana do projektu rękawica była zbyt gruba, powodując dyskomfort w użytkowaniu w szczególności przez dłuższy okres czasu. Początkowe kryterium elastyczności, przesądziło o wybraniu tej rękawicy, jednak cienka rękawica również spełniła by wymagania końcowego produktu. Rozwiązanie zastosowane w celu odczytu wygięcia palców z założenia wyglądało na idealnie pasujące do wymagań projektu. Pomimo swojej prostoty wykonania oraz braku pomiaru takich cech jak odwodzenie palców czy też wygięcie poszczególnych stawów, spełnia ono swoją podstawową funkcję. Problemem tego rozwiązania jest natomiast brak elastyczności sensorów. W momencie zgięcia palców droga od knykci do paznokci się wydłuża sprawiając że sensor jest poddawany sile nacisku od strony palca która jest tym spowodowana. Sensory te pomimo braku elastyczności są zbudowane z materiału wytrzymałego na rozciąganie dzięki czemu nie pękają podczas zgięciu palców, jednak w celu zapewnienia lepszego mocowania i większej ochrony, z jednej strony została przymocowana elastyczna guma która trzyma sensor przy czubkach palców, z drugiej natomiast sensor został wszyty w rękawicę. Problem który się pojawił w trakcie użytkowania pochodził ze sposobu wszycia sensora. Została do tego użyta nić przewodząca która z powodu rozciągliwości rękawicy nie mogła zostać wszyta na sztywno, w związku z czym stawiała ona mniejszy opór podczas zginania palców i niejako została wyciągnięta przez sensor, powodując tym brak dokładności odczytów. Nić ta oprócz niskiej elastyczności okazała się być nietrwała. W trakcie korzystania z rękawicy doszło do kilku pęknięć, które zostały ponownie związane, jednak została przerwana w ten sposób ciągłość obwodu. Przez dodanie dodatkowych wiązań odczyty z sensorów się pogorszyły, sprawiając że wygięcie palca wskazującego ma większy wpływ na odczyty z kciuka, niż zgięcie kciuka samo w sobie. Podobna sytuacja przytrafiła się z sensorem małego palca oraz serdecznego. Mała powierzchnia na dłoni wokół której należało poprawić wiele połączeń, sprawiła że część nici była blisko siebie, powodując momentami odczuwalne mrowienie na dłoni. Problem ten został rozwiązany poprzez zastosowanie izolacji od wewnętrznej strony rękawicy, jednak nie gwarantuje to przeciwdziałaniu zwarć w obwodzie. Konkludując, nić przewodząca nie jest najlepszym rozwiązaniem w celu połączenia elementów dla tego projektu i powinno zostać zastąpiona trwalszym połączeniem. Gdyby jednak została ona użyta, element przewodzący powinien znajdować się w środku oplotu, bądź powinny zostać zastosowane inne sposoby izolacji, a sama wytrzymałość nici powinna być znacznie większa. Mocowanie sensorów wygięcia powinno być bardziej trwałe oraz statyczne, nie pozwalając na przemieszczenie sensora na palcu. Alternatywą dla tego rozwiązania jest wykorzystanie czujników pomiaru wygięcia opartych o światło nadawane z jednej strony plastikowej tuby oraz miernika natężenia światła z drugiej. W ten sposób wiadomo że im mniejszy pomiar otrzymywanego światła, tym większe wygięcie tuby, której załamanie blokuje bezproblemowy dopływ światła. Rozwiązanie to również zapewnia pomiary niezniekształcone poprzez zachowanie innych sensorów a także wygląda na bardziej dokładne~\cite{ledsensor}.
 
 
  \section{Animacja modelu}
 \label{sec:iuAnimacja}
 Ostatnim elementem aplikacji dla projektu jest zapewnienie animacji dłoni. W tym celu został wykorzystany \textit{Google Sceneform}, dzięki któremu zaimportowano modele, ustawiono scenę, przypisano model a także obsługiwano przemieszczenie i orientację. Ostatnim brakującym elementem jest animacja modelu. Według dokumentacji starszej wersji projektu osiągnąć to można poprzez klasę \textit{SkeletonNode}, pozwalającą na dostęp do kości modelu, bez wykorzystania zewnętrznego programu graficznego. Jednak z niejasnych przyczyn klasa ta została usunięta w ostatniej wersji SDK, powodując brak możliwości wprowadzania zmian w modelu który został zaimportowany przy użyciu wtyczki. Problem ten rozwiązano poprzez wykorzystanie programu \textit{Blender}, dzięki któremu można było wyeksportować modele w wyznaczonej pozycji. Aby osiągnąć jednak animację modelu w czasie rzeczywistym, na podstawie dostępnych danych z sensorów wygięcia - cała klasa renderująca fragment~\ref{fig:ifceAnimacja} musi zostać napisana od nowa z wykorzystaniem innej technologii, ponieważ na oficjalnej stronie dystrybucji \textit{Sceneform}, jest napisane iż projekt został zarchiwizowany, w związku z czym taka opcja nie zostanie dodana~\cite{sceneform}. 
 
  \section{Błąd rotacji}
 \label{sec:iuRotacja}
 W przypadku rotacji jest wiele sposobów na polepszenie rezultatów. W prezentowanym projekcie wybrano podstawową metodę która wykorzystuje jedynie żyroskop oraz akcelerometr i przy ich użyciu wykorzystuje filtr komplementarny. Tak jak wcześniej wspomniano, aby dokładnie skorygować żyroskop na wszystkich trzech osiach, należy wykorzystać również magnetometr. Oprócz tego istnieje wiele filtrów takich jak Kalmana czy Madgwick'a które skutecznie usuwają szum, a także algorytmy wykorzystujące nowe pomiary w połączeniu z tymi zebranymi przed nimi. Możliwości łączenia technik udoskonalania odczytu rotacji z jednostek IMU sprawia, że nie ma jednego najlepszego rozwiązania, a ich wybór jest uzależniony on rodzaju projektu nad którym się pracuje~\cite{sensorslab}. 
 
 
  \section{Problem obliczania przesunięcia}
 \label{sec:iuPrzesunięcie}
 Obliczenie położenia kontrolera w przestrzeni, niewątpliwie należy do najtrudniejszego problemu w tym projekcie. Sedno problemu tkwi w niedokładności danych. Z powodu wykorzystania metody podwójnego całkowania, błąd uzyskiwany w cm przy pojedynczej całce, rośnie do m przy całkowaniu podwójnym. Ekran aplikacji jest mierzony w m, a ruch dłoni z kontrolerem ma ograniczony zasięg długości ramienia. Pomimo tego w niedużym czasie błąd rośnie do poziomu w którym model znika z ekranu użytkownika. Niestety nie istnieje łatwy sposób na skorygowanie błędów powstałych w wyniku tego algorytmu. Firmy zajmujące tym się tym problemem dodatkowo umieszczają czujniki pozwalające określić odbiornik urządzenia i ustalić pozycję względem niego, kamery zewnętrzne obserwujące ruch w przestrzeni a także dodatkowe czujniki optyczne. W przypadku rękawicy-kontrolera który może poruszać się we wszystkich kierunkach dodatkowy problem stanowi rotacja, przez którą błąd staję się coraz większy. W przypadku prostej aplikacji nie wykorzystującej zaawansowanych jednostek pomiarowych oraz algorytmów filtrujących, często efekt jaki można osiągnąć tą techniką nie sprawdza się w zastosowaniu, dlatego też dla tej aplikacji domyślnie funkcja ta jest wyłączona~\cite{sensorslab}. 
 
 