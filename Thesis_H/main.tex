\documentclass[inf, h]{pjatkThesis}
%
\usepackage{xurl}
\usepackage{float}
%\usepackage{fancyhdr}
%\pagestyle{fancy}
\usepackage{times}
\usepackage[polish,english]{babel}
%
\usepackage{sidecap}
\usepackage{graphicx}
\graphicspath{ {Images/} }
\usepackage{wrapfig}
\usepackage{subcaption}
\newcommand*{\captionsource}[2]{%
  \caption[{#1}]{%
    #1%
    \\\hspace{\linewidth}%
    {Źródło:} #2%
  }%
}
%
\usepackage{makeidx}
\usepackage{xargs} 
\usepackage{lipsum}
\usepackage[pdftex,dvipsnames]{xcolor}
%
%TODO Show \thiswillnotshow notes, remove line below for final version, and review after
%\setlength {\marginparwidth }{2cm}
\usepackage[disable,colorinlistoftodos,prependcaption,textsize=tiny]{todonotes} % use \usepackage[disable]{todonotes} to switch off
\newcommandx{\unsure}[2][1=]{\todo[linecolor=red,backgroundcolor=red!25,bordercolor=red,#1]{#2}}
\newcommandx{\info}[2][1=]{\todo[linecolor=OliveGreen,backgroundcolor=OliveGreen!25,bordercolor=OliveGreen,#1]{#2}}
\newcommandx{\change}[2][1=]{\todo[linecolor=blue,backgroundcolor=blue!25,bordercolor=blue,#1]{#2}}
\newcommandx{\improvement}[2][1=]{\todo[linecolor=Plum,backgroundcolor=Plum!25,bordercolor=Plum,#1]{#2}}
\newcommandx{\thiswillnotshow}[2][1=]{\todo[linecolor=blue,backgroundcolor=blue!25,bordercolor=blue,#1]{#2}}
%
%
\usepackage{listings,xcolor}
%Listings settings
\lstset{
tabsize = 4, 
showstringspaces = false, %% prevent space marking in strings, string is defined as the text that is generally printed directly to the console
numbers = left, 
commentstyle = \color{green}, 
keywordstyle = \color{blue}, 
stringstyle = \color{red}, 
rulecolor = \color{black}, %% set frame color to avoid being affected by text color
basicstyle = \small \ttfamily , %% set listing font and size
breaklines = true, 
numberstyle = \tiny,
}
%
%
%

\author{Kamil Kacprzak}
\album{s14004}
%\title{Rozwój technologii wirtualnej rzeczywistości na przykładzie rękawicy-kontrolera}
\title{Proces tworzenia manipulatora ręcznego dla potrzeb światów rzeczywistości wirtualnej}
\type{Praca inżynierska}
\supervisor{dr inż. Michał Tomaszewski}
\location{Warszawa}
\date{Czerwiec 2020}
%
%
%
\begin{document}
\selectlanguage{polish} 
\tableofcontents
%\listoftables

\begin{abstract}
W niniejszej pracy, zatytułowanej ``Analiza technologii rzeczywistości rozszerzonej w biznesie, ze szczególnym uwzględnieniem rękawicy-kontrolera" przedstawiono rozwój rzeczywistości rozszerzonej poczynając od pierwszych przedsięwzięć, rozpoczętych jeszcze przed początkami grafiki komputerowej, aż do chwili obecnej w której technologia ta jest wykorzystywana powszechnie w biznesie, redukując koszty i czas wielu pracowników. Szczegółowo opisano produkty rękawic-kontrolerów, które są powszechne na rynku i coraz częściej wykorzystywane w celu pogłębienia inercji. W rozdziale pierwszym pokazano jak dzieli się technologia rzeczywistości rozszerzonej a także co przyczyniło się do jej rozwoju. Następnie pokazano metody interakcji z kreowanymi rzeczywistościami. Bazując na tej wiedzy zostały przedstawione nowoczesne kontrolery, które korzystają jedynie z dłoni oraz opisano różne rodzaje występujących kontrolerów na rynku. W rozdziale~\ref{ch:biznes} pokazano jak biznes korzysta z tej technologii oraz jak może ją wykorzystać w przyszłości w celu osiągnięcia lepszych efektów. W ostatnim rozdziale został przedstawiony podstawowy projekt rękawicy, w celu lepszego zrozumienia wymaganych komponentów oraz budowy tego typu projektu. 

Słowa kluczowe: Rzeczywistość rozszerzona, Wirtualna rzeczywistość, Rozszerzona rzeczywistość, Mieszana rzeczywistość, Biznes, Rękawica, Kontroler
\newpage
\end{abstract}
\selectlanguage{english} 
\begin{abstract}
In this study, titled ``An analysis of extended reality in business particularly with regards to glove controllers”, a development of extended reality is presented. This study examines the said development, starting with the very first innovation initiated before the beginning of computer graphics, up until the present, when it is commonly used through the areas of business,  reducing time and cost expenses. This paper precisely focuses on virtual reality gloves, a segment in the business that is of high interest, especially due to the increased inertia regarding virtual reality experience. The first chapter addresses different types of extended reality, as well as key factors that had an effect on the development of this technology. In the following chapter, different methods of presentation and interaction with the environments in question are described. On the basis of the already discussed chapters, advanced controllers that are only using hands as a reference are presented and divided into categories according to the features used in those solutions. Chapter~\ref{ch:biznes} expresses a variety of ways in which XXI century businesses are exploiting the benefits of technology, and additionally presents predictions of future improvements, especially with regards to glove controllers. The final chapter discusses the creation of a basic glove controller, and as a result, a deeper understanding of the topic and components is obtained. 

Keywords: Extended reality, Virtual Reality, Augmented Reality, Mixed Reality, Business, Data glove, Vr Glove


\end{abstract}
 

\selectlanguage{polish} 
\pagenumbering{arabic}
\baselineskip=22pt
\chapter{Wstęp}
\label{ch:wstep}
\change{cel i zarys, wytłumaczenie tytułu}
\chapter{Zagadnienia}
\label{ch:zagadnienia}

\change{TODO: ch:zagadnienia}

\section{Czym jest wirtualna rzeczywistość?}
\label{sec:vr}
%
\section{Porównanie dostępnych środowisk do tworzenia aplikacji VR}
\label{sec:enviroment}
\section{Porównanie dostępnych SDK}
\label{sec:sdk}
%minus?
%
\section{Działanie kontrolerów w VR}
\label{sec:kontroleryVR}
%historia rekawica kontroler
	
	
\chapter{Komponenty rękawicy-kontrolera}
\label{ch:komponenty}
\change{Opis rękawic na rynku}

	\section{Nawigacja inercyjna}
	\label{sec:inercja}
	\change{INS - inertial navigation system}
	\change{Jak dostępne rękawice omijają problem}

	
	
	\section{IMU - Inercyjna jednostka pomiarowa}
	\label{sec:imu}
	
		\subsection{Żyroskop}
		\label{subsec:gyro}
		
		\subsection{Akcelerometr}
		\label{subsec:acc}
		
		\subsection{Magnetometr}
		\label{subsec:mag}
	
	\section{Stopnie swobody}
	\label{sec:swobody}	
	
\section{Porównanie Bluetooth z BLE(z ang. Bluetooth Low Energy)}
\label{sec:bvsble}
\improvement{TODO: BLE section}
\change{UUID section?}
tak zwany uniwersalny unikalny identyfikator znany pod akronimem UUID (z ang. Universally Unique IDentifier). 
	
	\section{Monitorowanie położenia palców}
	\label{sec:palce}
	
\chapter{Projekt rękawicy}
\label{ch:rekawica}
\chapter{Dedykowana aplikacja w systemie Android}
\label{ch:aplikacja}

\change{TODO}
\chapter{Dalszy rozwój projektu}
\label{ch:rozwoj}
\chapter{Podsumowanie}
\label{ch:podsumowanie}
W niniejszej pracy został pokazany projekt który bazując na dotychczasowych rezultatach na rynku rękawicy-kontrolera, wykorzystywanego w środowisku świata wirtualn
\addcontentsline{toc}{chapter}{Bibliografia}
%
%
%\addcontentsline{toc}{chapter}{Literatura}
%
\bibliographystyle{abbrv}
\bibliography{bibliografia}
\listoffigures
%   
\listoftodos[TODOs:]
%
\end{document}