\begin{abstract}
Tematem pracy jest "Rozwój technologii wirtualnej rzeczywistości na przykładzie rękawicy-kontrolera" w której pokazano czym jest świat wirtualny a także w jaki sposób kontroluje się w nim środowisko. Przedstawiono tematykę kontrolerów a także rękawicy-kontrolera, oraz ich podstawowych funkcjonalności. W rozdziale~\ref{ch:komponenty} została pokazane problemy oraz rozwiązania z jakimi się mierzą współczesne rozwiązania. Następnie pokazano własną implementacje rękawicy-kontrolera spełniającej podstawowe funkcje. Rękawica ta przesyła dane bezprzewodowo. Przykładowy odbiornik tych danych, w postaci aplikacji na Androida został pokazany w rozdziale~\ref{ch:aplikacja}. Rozdział ten porusza takie tematy jako implementacja łączności bluetooth, ustalenie orientacji, położenia a także animacja modelu dłoni na podstawie otrzymanych danych z kontrolera. Na podstawie pracy z projektem zostały wyciągnięte wnioski oraz możliwe ulepszenia projektu pokazane w rozdziale~\ref{ch:rozwoj}, po czym zostało sformułowane podsumowanie przeprowadzonego projektu, pokazujące trudności na jakie się napotkano w trakcie pracy. Problemy te należą do dziedziny nawigacji inercyjnej i nie istnieje uniwersalne rozwiązanie które można by zastosować. Celem pracy jest przeprowadzanie próby stworzenia własnego rozwiązania realizującego zagadnienia związane z tematyką kontrolerów - w tym przypadku prezentowanego jako rękawica.
\\
\\Słowa kluczowe: Wirtualna rzeczywistość, Kontroler,Rękawica, Nawigacja inercyjna, Akcelerometr, Żyroskop, Czujnik wygięcia
\newpage
\end{abstract}
\selectlanguage{english} 
\begin{abstract}
The topic of this thesis is "Development of virtual reality based on the example of glove controller", in which is shown what is a virtual reality and what are the ways of controlling and interacting with the universe in it. Controler topic, especially in the context of glove controllers, was addressed and it's functionality and usability cases were shown. In chapter~\ref{ch:komponenty} are shown challenges with which controller's designers are encountered. Next, the own implementation of the glove controller is presented, which is focused on delivering on basic functionality. Data are sent from the glove in a wireless manner. The receiver of this data, as Android application is described in chapter~\ref{ch:aplikacja}. The mentioned chapter discusses topics of Bluetooth Low energy implementation, the orientation of the device, it's displacement and model animation based on data received from the glove. Based on all of the experiences while working with this project, flaws and possible improvements were described in chapter~\ref{ch:rozwoj}, after which the summary and conclusions of this project were formed, displaying all of the difficulties encountered while creating glove proposal. Encountered issues belong to the immersive navigation realm, and there is no universal solution that could be applied. The goal of this paper is to take an attempt upon creating a homemade glove controller, which fulfills all of the basic requirements for a functional controller device.
\\
\\Key words: Virtual reality, Controller, Glove, Immersive navigation, Accelerometer, Gyroscope, Bend sensors 
\end{abstract}
 