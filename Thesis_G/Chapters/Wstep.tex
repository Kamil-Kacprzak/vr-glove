\chapter*{Wstęp}
\label{ch:wstep}
\addcontentsline{toc}{chapter}{Wstęp}
% Zastosowanie technologii rzeczywistości rozszerzonej w biznesie, ze szczególnym uwzględnieniem rękawicy-kontrolera
Zjawisko tworzenia sztucznej rzeczywistości jest znane ludziom od wielu dekad. Pierwsze prace w tym obszarze zostały podjęte jeszcze przed powstaniem grafiki komputerowej. Sam koncept wykorzystania w tej technologii dłoni jako kontrolera, powstał nie wiele później. W poniższych rozdziałach zostanie przedstawione jak technologia ta się  rozwinęła od tamtej pory, oraz jak liderzy rynku rozwiązują problemy które są im stawiane przez ograniczenia współczesnej technologii. Tematem pracy jest ``Zastosowanie technologii rzeczywistości rozszerzonej w biznesie, ze szczególnym uwzględnieniem rękawicy-kontrolera''. Przedstawiana praca pokazuję technologię rzeczywistości rozszerzonej oraz sposób w jaki jest ona prezentowana użytkownikom. Różne metody i podejścia sprawiają że rynek jest różnorodny oraz wypełniony wieloma rodzajami urządzeń które wspomagają realizm prezentowanych światów. Rozwój realizmu oraz wygody korzystania z tego rodzaju urządzeń sprawił że wiele firm coraz chętniej wykorzystuje część tej technologii w celu optymalizacji procesów oraz kosztów w biznesie. Temat ten jest szerokim zagadnieniem z powodu wielu rodzajów rzeczywistości jakie są wykorzystywane. W poniższej pracy zostaną opisane elementy składające się na termin rzeczywistości rozszerzonej oraz w jakiej dziedzinie poszczególne jej obszary mają przewagę nad innymi typami. Praca skupia się w szczególności na zastosowaniu technologii rękawic-kontrolerów w połączeniu z technologiami świata wirtualnego. Urządzenia te pozwalają na komunikację pomiędzy użytkownikiem a prezentowanymi światami w sposób realny oraz intuicyjny. W szczególności intuicyjność tego rozwiązania jest jednym z głównych powodów powstania tej pracy. Warto również zauważyć że rynek ten jest rynkiem rozwijającym się co zostanie pokazane na przykładzie rozwiązań już istniejących oraz skali zainteresowania tymi produktami. Produkty te zostaną szczegółowo opisane, ich podział ze względu na budowę oraz zastosowania które się różnią w zależności od implementowanych konceptów. Praca ma na celu przeanalizować podejścia na jakie twórcy tych produktów się zdecydowali oraz wady i zalety technologii z której korzystają. Głównym wyzwaniem w tym obszarze jest rozwiązanie problemów nawigacji inercyjnej, czyli rodzaju nawigacji pozwalającego na określenie położenia oraz orientacji w systemie, zazwyczaj bez układu odniesienia. Temat ten nie posiada uniwersalnego rozwiązania w związku z czym dodatkowe elementy są często wykorzystywane w celu usprawnienia funkcjonowania systemu. Oprócz informacji dotyczących położenia bardzo ważnym aspektem jest możliwość śledzenia palców dłoni. W ten sposób są spełnione podstawowe wymagania kontrolera, jednak aby wyróżnić się na rynku wiele firm prezentuje dodatkowe rozszerzenia wzmacniające odczucia podczas użytkowania, takie jak wibracje podczas kolizji w świecie wirtualnym, imitację dotyku przedmiotów czy blokadę położenia dłoni. Te rozszerzenia podstawowego produktu oraz wiele innych zostaną opisane w dalszej części tej pracy. W celu lepszego zrozumienia prezentowanych problemów, zostanie stworzony podstawowy projekt własnej rękawicy-kontrolera, dzięki czemu lepiej będzie można przyjrzeć się budowie takiego urządzenia, elementów wymaganych do jego stworzenia jak i również na podstawie tej pracy możliwość jego odtworzenia i zastosowania w domowych warunkach przy niewielkich kosztach.