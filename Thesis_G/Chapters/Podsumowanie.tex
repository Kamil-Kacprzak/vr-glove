\chapter*{Podsumowanie}
\label{ch:podsumowanie}
\addcontentsline{toc}{chapter}{Podsumowanie}
W powyższej pracy zaprezentowano rozwój rzeczywistości rozszerzonej, poczynając od pierwszych wynalazków które były skupione głównie na wirtualnej rzeczywistości, aż do czasów w których inne formy kreowanej rzeczywistości zostały stworzone. Pokazano różne inspiracje które przyczyniły się do rozwoju technologii jak i sposób w jaki jej rozwój przyczynił się do nowych wizji twórców. Pokazano jak prezentacja światów wirtualnych zmieniła się w czasie oraz jak urządzenia takie jak smartphone pozwoliły na łatwy dostęp do sztucznych obiektów w rzeczywistości prawie każdemu na wyciągnięcie ręki. Wraz z rozwojem metod prezentacji nastąpił również progres w interakcji z użytkownikiem, co sprawiło pogłębienie inercji w stworzony w ten sposób świat. Zastosowanie rozwiązań mieszanej rzeczywistości sprawia że granica pomiędzy tymi dwom światami zostaje powoli zacierana, a dostępność produktów sprawia że ich cena jest coraz bardziej przystępna dla przeciętnego użytkownika. Innym rodzajem urządzeń które próbują osiągnąć ten cel są kontrolery wykorzystujące dłoń użytkownika jako metodę interakcji. Metoda ta ciągle ma wiele ograniczeń, lecz wraz z biegiem lat coraz więcej firm stara się udoskonalić swoje produkty tak aby były coraz bardziej przystępne oraz uniwersalne. Pomimo swojej ceny zastosowanie tych produktów jest widoczne w biznesie gdzie inwestycja w tego rodzaju rozwiązanie technologiczne, często oprócz zakupu sprzętu dotyczy również stworzenie autorskich rozwiązań które pozwolą na osiągnięcie zakładanego celu takiego jak świadczenie lepszych usług czy szkolenie pracowników. W tej pracy pokazano w jaki sposób różne dziedziny biznesu wykorzystują te rozwiązania w celu usprawnienia swojej pracy. W przypadku lotnictwa częstym wykorzystaniem są symulatory które sprawiają że zarówno trening kadry jest ułatwiony jak i koszt związany z budową symulatorów znacząco spada. Treningi stają się bardziej wydajne a doświadczenia z odbycia tego rodzaju praktyki przynoszą lepsze rezultaty. W przypadku architektury, deweloperzy oraz właściciele nieruchomości wykorzystują nowe technologie w celu przyciągnięcia klientów oraz pokazania im możliwości jakie płyną z wyboru ich produktów - co by było nie możliwe do zrealizowania gdyby nie możliwość stworzenia sztucznej rzeczywistości którą na skinienie palca można dostosowywać do potrzeb danej osoby. Jednym z najważniejszych zastosowań jest usprawnianie możliwości lekarzy oraz tworzenie dla nich narzędzi dzięki którym mogą oni ratować ludzie życie. W tej sferze w szczególności wykorzystanie rękawic kontrolerów daje niewyobrażalne możliwości w szkoleniu jak i pomocy na odległość gdy lekarz specjalista będzie przy użyciu wirtualnych narzędzi pewnego dnia uratować realną osobę na odległość. Wszystkie te myśli i koncepty nie byłyby osiągalne gdyby nie rozrywka w której to nastąpiła popularyzacja rozwiązań związanych ze sztuczną rzeczywistością. Z roku na rok widać coraz większy wzrost urządzeń korzystających ze wszystkich zalet tego świata, a także twórców którzy wymyślają coraz to nowsze sposoby na przenoszenie kolejnych zmysłów do komputera. W celu dokładnego zrozumienia jak praca nad tymi produktami wygląda została zbudowana i zaprezentowana rękawica którą można zbudować i wykorzystać w dowolnych projektach przy nie dużych kosztach. Rozwój technologiczny rękawic kontrolerów na wielu przedstawianych przykładach potwierdza że oprócz wielu zalet płynących z ich wykorzystania, nadal pozostało do rozwiązania wiele problemów z którymi twórcy muszą się zmierzyć, w szczególności odczytów dotyczących położenia dłoni w przestrzeni. Różne formy rozwiązania tych problemów są stosowane jednak żadne nie jest idealne. Przewidywania specjalistów dotyczące rynku XR są obiecujące w związku z czym jest to idealny moment aby wprowadzić tego rodzaju rozwiązania do własnych firm i domów aby wspierać rozwój tej dziedziny jak i doświadczyć jej zalet. 